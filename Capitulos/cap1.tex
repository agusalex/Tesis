\chapter{Introducción}

\label{Capitulo_1} % Para referenciar el capítulo en el documento, usar \ref{Capitulo_1}
\section{Motivación y Antecedentes}
En el contexto de la planificación urbana y las ciudades inteligentes o \textit{Smart Cities}, el problema de encontrar espacios de estacionamiento es algo que se encuentra en estado de plena investigación y desarrollo. Se consultó la literatura y se encontró soluciones que utilizan sensores colocados en el pavimento o numerosas cámaras más algún tipo de software que permite detección de estacionamiento. Estas soluciones, si bien pueden ser efectivas, tienen como problema el uso de hardware costoso con gran poder de cómputo, cuya instalación requiere de gran esfuerzo y recursos.

Basándose en la experiencia previa adquirida durante la colaboración con el municipio de San Miguel en la investigación y desarrollo del proyecto \textit{Smart Cities} \cite{sedici89263}, el grupo de investigación en redes AD-HOC de la UNGS, se puso como objetivo el desarrollo de una solución de bajo costo para \textit{Smart Parking}.

Luego de analizar el estado del arte se concluyó que existía novedad y utilidad en un sistema \textit{wireless} de rastreo de vehículos y transeúntes. Este podría permitir a su vez en trabajos futuros identificar si un vehículo se encuentra en movimiento, detenido o ha estacionado.

Un sistema completo como el que el equipo de investigación se propone analizar, permitiría detectar individuos que posean dispositivos móviles o cualquier otro tipo de dispositivo que posea \acs{wifi} o incluso \acs{ble}. Una vez identificado, el sistema podrá ser capaz de detectar su ubicación a lo largo del tiempo y finalmente predecir si dicho dispositivo se encuentra dentro de un vehículo en movimiento, o es un transeúnte caminando, o bien es un vehículo que acaba de estacionar y a continuación marcar esa plaza como ocupada o detectar una zona que posee embotellamiento. Esta información es crucial para la administración de una ciudad que intenta introducirse al paradigma de \textit{Smart City} ya que le permitiría actuar de manera acorde ante estos eventos.

Por ejemplo, utilizando una aplicación de \textit{Smart Parking} sugerir al usuario dónde estacionar. Lo que es más, hasta podría permitirle a la ciudad, mediante el uso de dicha aplicación, ajustar las tarifas de estacionamiento según cuán congestionada se encuentre la zona. De esta manera permitiendo ahorro de combustible a los vehículos al no tener que recorrer mucha distancia para encontrar una plaza de estacionamiento y poder descomprimir los centros urbanos en horas de alta demanda.

Para que un dispositivo cliente se conecte a una red Wi-Fi, el primer paso es identificar los puntos de acceso (\acs{ap}) disponibles dentro de su rango. Este proceso utiliza dos tipos principales de tramas: los \acl{beacons}, enviados periódicamente por los \acs{ap}, y las \acl{prq}, que los clientes transmiten activamente cuando necesitan más información sobre las redes cercanas.

Los \acs{beacons} son señales que los \acs{ap} emiten a intervalos regulares, generalmente cada 100 ms, para anunciar la existencia de la red. Estas tramas incluyen datos esenciales como el nombre de la red (\acs{ssid}), la dirección MAC del punto de acceso (\acs{bssid}) y detalles técnicos sobre los canales, velocidades y métodos de cifrado soportados, como \acs{wpa2} o \acs{wpa3}. Este método permite a los dispositivos cliente identificar redes disponibles de manera pasiva, simplemente escuchando, lo cual es eficiente en términos de energía porque no implica que el cliente envíe datos.

En contraste, las \acl{prq} son tramas activas que los dispositivos cliente utilizan cuando necesitan información más específica o están buscando una red particular. Al enviar estas solicitudes, el cliente puede recibir respuestas de los \acs{ap} en el área, lo que resulta útil en escenarios como el roaming entre diferentes puntos de acceso o cuando el escaneo pasivo no es suficiente debido a interferencias o congestión.

Estos dos métodos, el escaneo pasivo mediante \acl{beacons} y el escaneo activo con \acl{prq}, se utilizan en conjunto para garantizar la detección y conexión a redes. Los \acl{beacons} reducen el consumo de energía al no requerir interacción activa, mientras que los \acl{prq} se adaptan a situaciones donde se requiere intervención activa. De esta manera, una vez descubierto el \acs{ap} se puede proceder con los siguientes pasos de la conexión wifi.


\section{Objetivo y alcance}
El presente trabajo investiga la viabilidad de estimar la ubicación y el recorrido de dispositivos móviles de manera pasiva, sin necesidad de que estos se encuentren conectados a una red \acs{wifi}. Para ello, se propone el uso de sensores de bajo costo basados en \acs{esp}, capaces de captar tramas de control 802.11 como los paquetes \acl{prq} y mediante el análisis del nivel de señal \acs{rssi} recibido en múltiples puntos, estimar la posición del dispositivo utilizando técnicas de multilateración. El sistema desarrollado demuestra que es posible realizar seguimiento pasivo de dispositivos móviles con precisión aceptable, lo cual abre la puerta a futuras aplicaciones de monitoreo urbano en entornos de \textit{Smart Cities}.

Como parte de esta línea de trabajo, se desarrolló un sistema funcional de localización pasiva que permite estimar la ubicación de dispositivos móviles sin requerir conexión previa. Además del sistema mismo, se desarrollaron una serie de herramientas complementarias publicadas como software libre, incluyendo firmware personalizado para los dispositivos embebidos utilizados, bibliotecas de multilateración y entornos de simulación. Estos recursos permiten tanto la reproducción de los experimentos como su adaptación a distintos escenarios urbanos.

Si bien el objetivo final del equipo de investigación es poder determinar con certeza el recorrido que realiza un vehículo al buscar estacionamiento en una ciudad.
Este trabajo se limita al objetivo investigar si existe la posibilidad de identificar el comportamiento de transeúntes sin necesidad de una conexión previa realizada por el usuario. Más precisamente nos planteamos ver cuán factible es, utilizando dispositivos de bajo costo, estimar la posición y recorrido de un teléfono móvil transportado por un individuo sin necesidad de que este se encuentre conectado, es decir, de manera pasiva. El seguimiento móvil mediante \acs{rssi} o \acl{rssi} y sensores \acs{esp} se planteó como una solución de bajo costo para determinar la ubicación de un dispositivo móvil.

Para tener resultados consistentes y poder saber a ciencia cierta si nuestro sistema era viable y poder plantear experimentos sencillos. Se tomó la decisión de realizar la conexión con el dispositivo móvil de manera diferente a en el escenario ideal donde el dueño del dispositivo no realiza ninguna modificación en la configuración de este y aún así es posible rastrearlo. La modificación realizada, fue encender el modo \acs{ap} \acs{wifi} del teléfono. Esto nos permite aumentar la frecuencia de envío de paquetes de tipo \acl{prq}. Exploraremos el motivo y los mecanismos más adelante, sin embargo este mismo sistema puede funcionar sin intervención del dispositivo rastreado si se quisiera aunque con menor frecuencia de envío.
\newline
A continuación, describimos la implementación de un sistema de seguimiento móvil que utiliza el \acs{rssi} recibido a partir de microcontroladores \acs{esp}, para inferir información de ubicación de un dispositivo móvil mediante paquetes de control 802.11. El sistema utiliza una red de puntos de referencia fijos, cada uno equipado con un \acs{esp}, para medir la intensidad de las señales recibidas del dispositivo móvil y calcular su ubicación utilizando algoritmos de multilateración. El bajo coste y la amplia disponibilidad de dichos sensores, así como su diseño de bajo consumo, los hacen muy adecuados para su uso en aplicaciones de seguimiento de localización. Demostraremos la eficacia del sistema de seguimiento móvil mediante experimentos que evalúan su precisión.

\section{Estructura del trabajo}
Este trabajo se organiza de la siguiente manera: En el Capítulo \ref{Capitulo_1} se plantea la introducción. En el Capítulo \ref{Capitulo_2}, se realiza una revisión de la literatura sobre las tecnologías inalámbricas y las técnicas de posicionamiento. En el Capítulo \ref{ExperimentoSimulado} se describe un experimento simulado de perfilado y multilateración. El Capítulo \ref{ExperimentoReal} presenta un experimento de campo real, para el perfilado de señales \acs{wifi} y multilateración. En el Capítulo \ref{Cap_Errores}, se discuten las fuentes de errores y las posibles mejoras. Finalmente, en el Capítulo \ref{Capitulo_Conclusiones} se presentan las conclusiones y las recomendaciones para futuros trabajos. Los Apéndices contienen detalles adicionales sobre los experimentos y los resultados.