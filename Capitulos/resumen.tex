Existen múltiples organizaciones, como centros comerciales, que proveen Wi-Fi gratuito como incentivo para luego, aprovechando esa conexión, poder determinar el comportamiento de sus clientes; que negocios prefieren, cuánto tiempo pasan en el establecimiento y hasta que gondolas les interesa más. Esto requiere sin embargo de que los individuos decidan conectarse a dicho Wi-Fi\cite{Jordan2015ReliablePD}.
\newline
Nuestro trabajo consiste en investigar si existe la posibilidad de deducir el recorrido de transeúntes y vehículos usando un método similar sin necesidad de una conexión previa realizada por el usuario. Mas precisamente nos planteamos ver cuán factible es, utilizando dispositivos de bajo costo, estimar la posición y recorrido de un teléfono móvil sin necesidad de que este se encuentre conectado a una red Wi-Fi, es decir, de manera pasiva.
\newline
En este trabajo, describimos la implementación de un sistema de seguimiento que utiliza el RSSI (Received Signal Strength Indication) recibido a partir de sensores ESP8266, para inferir información de ubicación de un dispositivo móvil mediante paquetes de control 802.11. El sistema utiliza una red de puntos de referencia fijos, cada uno equipado con un sensor ESP8266, para medir la intensidad de las señales recibidas del dispositivo móvil y calcular su ubicación utilizando algoritmos de  multilateración. El bajo coste y la amplia disponibilidad de dichos sensores, así como su diseño de bajo consumo, los hacen muy adecuados para el despliegue masivo de una red de sensores urbana en particular para su uso en aplicaciones de rastreo de localización.
\newline
Como resultado, se logró desarrollar un sistema funcional de localización pasiva con pruebas exitosas en escenarios reales y simulados, junto con diversas herramientas de software publicadas. Incluyendo bibliotecas de multilateración, firmware personalizado, y entornos de simulación. Estas aydan a demostrar la viabilidad técnica de este enfoque para aplicaciones urbanas en el contexto de \textit{Smart Cities} y en particular el monitoreo del flujo urbano.